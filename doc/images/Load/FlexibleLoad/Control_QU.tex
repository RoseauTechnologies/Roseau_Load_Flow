\input{Preamble}%

\usepackage{pgfplots}%
\pgfplotsset{compat=newest}%
\usepgfplotslibrary{groupplots, colorbrewer}%

\begin{document}
\begin{tikzpicture}[%
    show background rectangle,%
    tight background,%
    background rectangle/.style={fill=white}%
  ]
  %
  % Common parameters
  %
  \pgfmathsetmacro{\umaxvaleur}{250.0}%
  \pgfmathsetmacro{\umaxnormvaleur}{1.0}%
  \pgfmathsetmacro{\uupvaleur}{240.0}%
  \pgfmathsetmacro{\uupnormvaleur}{\uupvaleur/\umaxvaleur}%
  \pgfmathsetmacro{\udownvaleur}{220.0}%
  \pgfmathsetmacro{\udownnormvaleur}{\udownvaleur/\umaxvaleur}%
  \pgfmathsetmacro{\uminvaleur}{210.0}%
  \pgfmathsetmacro{\uminnormvaleur}{\uminvaleur/\umaxvaleur}%
  \pgfmathsetmacro{\unomvaleur}{(\udownvaleur+\uupvaleur)/2.0}%
  \pgfmathsetmacro{\unomnormvaleur}{\unomvaleur/\umaxvaleur}%
  \pgfmathsetmacro{\umidminvaleur}{(\udownvaleur+\uminvaleur)/2.0}%
  \pgfmathsetmacro{\umidminnormvaleur}{\umidminvaleur/\umaxvaleur}%
  \pgfmathsetmacro{\umidmaxvaleur}{(\uupvaleur+\umaxvaleur)/2.0}%
  \pgfmathsetmacro{\umidmaxnormvaleur}{\umidmaxvaleur/\umaxvaleur}%

  \pgfmathsetmacro{\xminvaleur}{\uminvaleur - 2.5}%
  \pgfmathsetmacro{\xminnormvaleur}{\xminvaleur/\umaxvaleur}%
  \pgfmathsetmacro{\xmaxvaleur}{\umaxvaleur + 2.5}%
  \pgfmathsetmacro{\xmaxnormvaleur}{\xmaxvaleur/\umaxvaleur}%

  \pgfmathsetmacro{\yminnormvaleur}{-1}%
  \pgfmathsetmacro{\ymaxnormvaleur}{1}%

  \pgfmathsetmacro{\qthnormvaleur}{0.30}%

  %
  % Style
  %
  \tikzset{lisse/.style={line width=0.3mm, domain=\xminnormvaleur:\xmaxnormvaleur, samples=75,
      mark=none}}%
  \tikzset{non lisse/.style={line width=0.3mm, mark=*}}%

  \begin{axis}[%
      height=7cm,%
      width=0.9\textwidth,%
      enlarge y limits,%
      grid=major,%
      xlabel={$|V_{p_1}-V_{p_2}|$},%
      xtick={\uminnormvaleur,\umidminnormvaleur,\udownnormvaleur,\unomnormvaleur,\uupnormvaleur,\umidmaxnormvaleur,\umaxnormvaleur},%
      xticklabels={%
        $\uminnorm$,,$\udownnorm$,$\unomnorm$,$\uupnorm$,,$\umaxnorm$%
      },%
      y tick label style={%
        /pgf/number format/.cd,%
        set thousands separator={},%
        fixed,%
        fixed zerofill,%
        precision=1,%
        use comma%
      },%
      ytick={\yminnormvaleur,0,\qthnormvaleur,\ymaxnormvaleur},%
      yticklabels={-1,0,$\dfrac{Q^{\theo}_{p_1p_2}}{\smax_{p_1p_2}}$,1},%
      xmin=\xminnormvaleur,%
      xmax=\xmaxnormvaleur,%
      ymin=\yminnormvaleur,%
      ymax=\ymaxnormvaleur,%
      legend columns=2,%
      legend style={%
        at={(0.5,-0.25)},%
        anchor=north,%
        nodes={text width=4cm}%
      },%
      cycle list/YlOrRd-5,%
      cycle list name=YlOrRd-5%
    ]%%%%%%% initialize YlOrRd-5%

    % Fonction linéaire par morceaux
    \addplot[non lisse, red] coordinates {%
        (\xminnormvaleur,-1)%
        (\uminnormvaleur,-1)%
        (\udownnormvaleur,\qthnormvaleur)%
        (\uupnormvaleur,\qthnormvaleur)%
        (\umaxnormvaleur,1)%
        (\xmaxnormvaleur,1)%
      };%
    \addlegendentry{Non-smooth control};%

    % Soft clipping functions
    \pgfplotsset{cycle list shift=-1}% Reset cycle to 0
    \foreach \alphavaleur in {50,100,200,300,400} {%
        \addplot+[lisse] expression {%
            \qthnormvaleur + ( 1.0/(\alphavaleur*(\uminnormvaleur-\udownnormvaleur)) *
            ln((1+exp(\alphavaleur*(x-\udownnormvaleur)))/(1+exp(\alphavaleur*(x-\uminnormvaleur))))-1.0
            )*(1+\qthnormvaleur) + ( 1.0/(\alphavaleur*(\umaxnormvaleur-\uupnormvaleur)) *
            ln((1+exp(\alphavaleur*(x-\uupnormvaleur)))/(1+exp(\alphavaleur*(x-\umaxnormvaleur))))
            )*(1-\qthnormvaleur) };%
        \addlegendentryexpanded{Soft clipping ($\alpha=\num{\alphavaleur}$)};%
      };%
  \end{axis}
\end{tikzpicture}
\end{document}
% Local Variables:
% mode: latex
% TeX-engine: luatex
% TeX-source-correlate-method-active: synctex
% ispell-local-dictionary: "british"
% coding: utf-8
% LaTeX-indent-level: 2
% fill-column: 120
% End:
