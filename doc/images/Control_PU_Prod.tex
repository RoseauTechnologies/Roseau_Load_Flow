\documentclass[%
crop,%
tikz,%
convert={outext=.svg,command=\unexpanded{pdf2svg \infile\space../_static/\outfile}},%
multi=false%
]{standalone}%
\usepackage[utf8]{luainputenc}%
\usepackage{amsmath}%
\usepackage{roseautechnologies}%
\usepackage{pgfplots}%
\pgfplotsset{compat=newest}%
\usepgfplotslibrary{groupplots, colorbrewer}%
\usetikzlibrary{backgrounds}%
\input{Definitions.tex}%
\begin{document}
\begin{tikzpicture}[%
    show background rectangle,%
    tight background,%
    background rectangle/.style={fill=white}%
    ]
    %
    % Common parameters
    %
    \pgfmathsetmacro{\umaxvaleur}{250.0}%
    \pgfmathsetmacro{\umaxnormvaleur}{1.0}%
    \pgfmathsetmacro{\uupvaleur}{240.0}%
    \pgfmathsetmacro{\uupnormvaleur}{\uupvaleur/\umaxvaleur}%
    \pgfmathsetmacro{\udownvaleur}{220.0}%
    \pgfmathsetmacro{\udownnormvaleur}{\udownvaleur/\umaxvaleur}%
    \pgfmathsetmacro{\uminvaleur}{210.0}%
    \pgfmathsetmacro{\uminnormvaleur}{\uminvaleur/\umaxvaleur}%
    \pgfmathsetmacro{\unomvaleur}{(\udownvaleur+\uupvaleur)/2.0}%
    \pgfmathsetmacro{\unomnormvaleur}{\unomvaleur/\umaxvaleur}%
    \pgfmathsetmacro{\umidminvaleur}{(\udownvaleur+\uminvaleur)/2.0}%
    \pgfmathsetmacro{\umidminnormvaleur}{\umidminvaleur/\umaxvaleur}%
    \pgfmathsetmacro{\umidmaxvaleur}{(\uupvaleur+\umaxvaleur)/2.0}%
    \pgfmathsetmacro{\umidmaxnormvaleur}{\umidmaxvaleur/\umaxvaleur}%

    \pgfmathsetmacro{\xminvaleur}{\uminvaleur - 2.5}%
    \pgfmathsetmacro{\xminnormvaleur}{\xminvaleur/\umaxvaleur}%
    \pgfmathsetmacro{\xmaxvaleur}{\umaxvaleur + 2.5}%
    \pgfmathsetmacro{\xmaxnormvaleur}{\xmaxvaleur/\umaxvaleur}%

    \pgfmathsetmacro{\yminnormvaleur}{0}%
    \pgfmathsetmacro{\ymaxnormvaleur}{1}%

    %
    % Style
    %
    \tikzset{lisse/.style={line width=0.3mm, domain=\xminnormvaleur:\xmaxnormvaleur, samples=75,
            mark=none}}%
    \tikzset{non lisse/.style={line width=0.3mm, mark=*}}%

    \begin{axis}[%
        height=7cm,%
        width=0.9\textwidth,%
        enlarge y limits,%
        grid=major,%
        xlabel={$|V_{\ell,p_1}-V_{\ell,p_2}|$},%
        xtick={\uminnormvaleur,\umidminnormvaleur,\udownnormvaleur,\unomnormvaleur,\uupnormvaleur,\umidmaxnormvaleur,\umaxnormvaleur},%
        xticklabels={%
            $\uminnorm$,,$\udownnorm$,$\unomnorm$,$\uupnorm$,,$\umaxnorm$%
        },%
        y tick label style={/pgf/number format/.cd,%
            set thousands separator={},%
            fixed,%
            % fixed zerofill,%
            precision=1,%
            use comma%
        },%
        xmin=\xminnormvaleur,%
        xmax=\xmaxnormvaleur,%
        ymin=\yminnormvaleur,%
        ymax=\ymaxnormvaleur,%
        legend columns=2,%
        legend style={%
            at={(0.5,-0.25)},%
            anchor=north,%
            nodes={text width=4cm}%
        },%
        cycle list/YlOrRd-5, % initialize YlOrRd-5
        cycle list name=YlOrRd-5%
        ]

        % Piecewise linear function
        \addplot[non lisse, red] coordinates {%
            (\xminnormvaleur,1)%
            (\uupnormvaleur,1)%
            (\umaxnormvaleur,0)%
            (\xmaxnormvaleur,0)%
        };%
        \addlegendentry{Non-smooth control};%


        % Soft clipping functions
        \pgfplotsset{cycle list shift=-1}% Reset cycle to 0
        \foreach \alphavaleur in {50,100,200,300,400} {%
            \addplot+[lisse] expression {%
                1.0 + 1.0/(\alphavaleur*(\umaxnormvaleur-\uupnormvaleur)) *
                ln((1+exp(\alphavaleur*(x-\umaxnormvaleur)))/(1+exp(\alphavaleur*(x-\uupnormvaleur))))
            };%
            \addlegendentryexpanded{Soft clipping ($\alpha=\num{\alphavaleur}$)};%
        };%
    \end{axis}
\end{tikzpicture}
\end{document}
% Local Variables:
% mode: latex
% TeX-engine: luatex
% TeX-source-correlate-method-active: synctex
% ispell-local-dictionary: "british"
% coding: utf-8
% LaTeX-indent-level: 4
% fill-column: 100
% End:
